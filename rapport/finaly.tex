
Voici une liste, plus ou moins exhaustive, des améliorations
possibles, voir souhaitables, du prototype proposé.


\subsection{Amélioration de la fiabilité}

\begin{itemize}
\item[] Des tests à grande échelle sont à faire pour pouvoir détecter
  d'éventuels problèmes dans la production. 
\item[] Des étapes de traitement spécifique au fournisseur sont à
  faire (ex: lignes roses d'Avionics).
\item[] Le cas spécial des SOUNDG est à traiter dans le
  \textit{DisplayStyler}.
\item[] Implémenter la surveillance des threads de calcul.
\item[] Implémenter les log du \textit{DisplayStyler}.
\item[] Gestion des noms en UTF16 avec Ogr.
\end{itemize}

\subsection{Amélioration des performances}

\begin{itemize}
\item[] Remplacer toutes les communications synchrones par des
  communications non-bloquantes/asynchrones.
\item[] Faire du recouvrement calcul/communication.
\item[] Optimiser l'écriture en BD, si possible.
\item[] Lancer les Worker role dès le premier fichier s57 uploadé.
\item[] S'intéresser aux index des BD.
\item[] Implémenter un deuxième worker role pour la copie (taille de
  machine différente).
\item[] S'intéresser aux tables azure (non SQL).
\end{itemize}

\subsection{Amélioration de l'expérience opérateur}

\begin{itemize}
\item[] Implémenter la chaine de production en entier (tables
  CATALOGUE\_RCL\_ALL, ...)
\item[] Enchainer les étapes de manière automatique entre les
  post-productions.
\item[] Implémenter une interface spécifique (Web, SSIS, ...).
\item[] Automatiser l'interruption d'une production.
\end{itemize}



%\subsection{Amélioration}
%%\subsection{Historique} %% ??
\section{Bilan}
Ce premier prototype prouve que la réalisation de cartes vectorielles,
pour le logiciel MaxSea Time Zero, en environnement cloud est
possible.  La faisabilité étant démontrée, il reste à examiner les
facteurs humains et économiques pour décider de la réalisation d'une
production plus aboutie.\\

Ce prototype veut résoudre le problème du manque de temps par la
solution du cloud computing. D'autres solutions comme la
parallèlisation à grain fin pour une production en local nécessitant
moins de ressources temporelles, ou encore une version hybride couplée
à une offre de PaaS, n'ont pas été envisagées.


% Remerciments ?
