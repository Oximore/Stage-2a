
Voici une liste, plus ou moins exaustive, des améliorations possibles,
voir souhaitables, du prototype proposé.


\subsection{Amélioration de la fiabilité}

\begin{itemize}
\item[] Des tests à grande échelle sont à faire pour pouvoir détecter
  d'éventuels problème dans la production. 
\item[] Des étapes de traitement spécifiques au fournisseurs sont à
  faire (ex: lignes roses d'avionics).
\item[] Le cas spécial des SOUNDG est à traiter dans le
  \textit{DisplayStyler}.
\item[] Implémenter la surveillance des threads de calcul.
\item[] Implémenter les log du \textit{DisplayStyler}.
\item[] Gestion des nom en UTF16 avec Ogr.
\end{itemize}

\subsection{Amélioration des performances}


\begin{itemize}
\item[] Remplacer toutes les communications synchrone par des
  communications non-bloquantes/asynchrones.
\item[] Faire du recouvrement calcul/communication.
\item[] Optimiser l'écriture en BD, si possible.
\item[] Lancer les Worker role dès le premier fichier s57 uploadé.
\item[] S'interser aux index des BD.
\item[] Implémenter un deuxième worker role pour la copie (taille de
  machine différente).
\item[] S'intereser aux tables azure (non SQL).
\end{itemize}

\subsection{Amélioration de l'éxpérience opérateur}
\begin{itemize}
\item[] Implémenter la chaine de production en entier (tables
  CATALOGUE\_RCL\_ALL, ...)
\item[] Enchainer les étapes de manière automatique entre les
  post-productions.
\item[] Implémenter une interface spécifique (Web, SSIS, ...).
\item[] Automatiser l'interuption d'une production.
\end{itemize}



%\subsection{Amélioration}
%%\subsection{Historique} %% ??
\section{Conclusion}
% Remerciments ?
