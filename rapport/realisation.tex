%%%%%%%%%%%%%%%%%%%%%%%%%%%%%%%%%%%%%%%%%%%%%%%%%%%%%%%%%%%%%%%%%%%%%%%
%
% Lignes de conduites
%
%%%%%%%%%%%%%%%%%%%%%%%%%%%%%%%%%%%%%%%%%%%%%%%%%%%%%%%%%%%%%%%%%%%%%%%

\subsection{Lignes de conduites}

Les pistes explorées par Ronan GOLHEN mettent en avant l'offre de
\textit{PaaS}\,\footnote{\textit{Plateform as a Service}, Plateforme
  en tant que service} Azure de \textit{Microsoft}. Les
caractéristiques principales de cette offre sont de libérer la société
de la gestion des serveurs et de ne payer qu'à l'usage. Sa philosophie
incite fortement aux modèles de calcul parallèle. 
%en effet on paye ici à l'utilisation, et non pour une période comme
%avec l'entretient de machines


Une architecture logicielle permettant de traiter une partie ou
l'ensemble de la production vecteur à l'intérieur du Cloud
\textit{Microsoft} doit être mise en place. Ce travail devra être
exploitable par les opérateurs de \textit{MapMedia} sans apprentissage
des technologies de developpement.


%- moyens à mettre en œuvre.
%\subsection{Moyens à mettre en œuvre}
Migrer la chaine de production demande une bonne connaissance des
principes de workflow, de développement Azure et du Framework métier
\textit{OGR}.\\

 Le workflow est à reconsidérer, on ne peut pas juste copier la chaine
 existante pour deux raisons principales.

Tout d'abord le service hébergeur est distant et on ne peut pas se
permettre de controler toute la production dans les locaux, ce qui
impliquerait un rapatriment des données entre chacune de ces étapes
(plusieurs Giga-octets). 

Ensuite le fait que la chaine utilise des logiciels propriétaires pose
un problème de licence car ils sont liés à une machine physique,
principe à bannir en terme de technologie cloud.  Nous avons contacté
SAFE SOFTWARE, la société qui développe FME, cependant celle ci n'est
pas près à fournir des licences payantes au nombre
d'utilisateurs. Leur principe de licence prévoit que leur logiciel
doit s'exécuter toujours sur la même machine, chose inenvisageable en
univers cloud.\\


Azure possède un Framework particulier qu'il faut maitriser. Par
exemple les espaces de stockage permanent sont séparés des ressources
physiques de la machine, qui elles, deviennent des ressources
temporaires. Ainsi à chaque étape de calcul on sauvegarde les
résultats dans des espaces de stockage persistant appelé Azure Blob.\\

 \textit{OGR} sera la principale alternative à FME. En effet on doit
 pouvoir se passer de ce logiciel propriétaire et on peut se servir du
 Framework qu'il utilise puisque ce dernier est open source. Il a donc
 fallut recoder les étapes de production sans passer par
 l'intermédiaire de FME.\\



%%%%%%%%%%%%%%%%%%%%%%%%%%%%%%%%%%%%%%%%%%%%%%%%%%%%%%%%%%%%%%%%%%%%%%%
%
%   Archi globale ??
%
%%%%%%%%%%%%%%%%%%%%%%%%%%%%%%%%%%%%%%%%%%%%%%%%%%%%%%%%%%%%%%%%%%%%%%%

\subsection{}

% process générale pour produire des cartes



\subsubsection{}

\subsubsection{}

\subsubsection{}









%%%%%%%%%%%%%%%%%%%%%%%%%%%%%%%%%%%%%%%%%%%%%%%%%%%%%%%%%%%%%%%%%%%%%%%
%
%   Plateforme Azure
%
%%%%%%%%%%%%%%%%%%%%%%%%%%%%%%%%%%%%%%%%%%%%%%%%%%%%%%%%%%%%%%%%%%%%%%%


%% Plus en Annexe ?

\subsection{Services fournit par la plateforme \textit{Windows Azure}}
Cette partie a pour but de familiariser le lecteur avec les conceptes
que nous emploirons plus tard. Nous allons évoquer des services
proposés par Microsoft.\\


\subsubsection{Déploiement Azure}
L'offre de \textit{PaaS} Azure s'appuit principalement sur le fait de
déployer facilement des logiciels. Pour ce faire il faut développer un
code qui serat zippé dans un fichier de binaire et ressources
(extention \textit{.cspkg}) et un fichier de configuration (extention
\textit{.cscfg}). Un fois ces deux fichiers uploadés, via le protail
Azure\,\footnote{\textit{https://manage.windowsazure.com/}}, on peut
gérer notre déployement. \\

\subsubsection{Worker Roles}
La cellule de base de l'architecture logicielle aportée est le Worker
Role, autrement dit le rôle de travail. \\

Lorsqu'on parle de worker role on fait souvent référence au code qui
le compose. L'unité de travail dans le déployement Azure, qui
correspond à ce worker role, est une instance de ce worker role.  On
peut faire ici un parallèle avec un objet et une classe en
POO\,\footnote{Programmation Orientée Objet}.\\


Une implémentation de cette fonctionnalitée est fournit par
l'API\,\footnote{\textit{http://msdn.microsoft.com/en-us/library/microsoft.windowsazure.serviceruntime.roleentrypoint\_members.aspx}}. Il
sagit d'une classe, C\# dans notre cas, nommée RoleEntryPoint. On peut
créer une classe héritant de celle ci et redéfinir les méthodes
$OnStart()$, $Run()$ et $OnStop()$. La métheode la plus importante est
la méthode $Run()$, on y met le code que l'on veut exécuter.\\

Un worker role est cencé tourner en boucle, ceci sans jamais
s'arrêter. Le processus peut tout de même s'intérompre dans des cas
particuliers. Si l'utilisateur décide de stopper son déployement ou si
le rôle est transférer sur une autre machine alors la méthode
$OnStop()$ est appelée. Dans le cas d'un bug, bug du programme comme
la monopolisations de toutes les ressources ou bug externe comme une
coupure de courant, la méthode $OnStop()$ ne peut être appelée.\\

Le processus correspondant à une instance d'un worker role est exécuté
sur une VM\,\footnote{\textit{Virtual Machine}, Machine Virtuelle}
spécialement montée pour lui. On dispose sur cette machine, en
fonction des options de configuration, d'un certain espace mémoire et
d'une certaine quantité de disque. Les données enregistrées sur le
disque sont effacées avec l'intéruption de la VM, lors de l'arret de
l'instance ou d'un disfonctionnement du serveur. De plus ces données
sont local à l'instance, les autres instances ne peuvent y accéder.\\


Pour pouvoir sauvegarder des résultats accéssibles par tous, autres
instances d'un même déploiement et utilisateurs extérieurs, la
plateforme Azure propose deux outils, l'Azure Storage et Azure SQL.
On peut bien sur coder nos propores outils, qui upload les données des
instances sur un serveur de l'entreprise par exemple. Cependant ces
outils ont deux avantages majeurs. Ils sont déjà fait et sont situé
sur des machines proches des VM, d'un point de vu géographique, ce qui
augement considérablement la bande passante.\\


\subsubsection{Azure Storage}
Azure Storage est un espace de stockage de données numériques de
grande capacité. On y trois services différents, les blobs, les queues et les tables.\\

Les blobs sont des objets binaires non structurés, on peut les
assimiler à des fichiers. Les queues sont des files
FIFO\,\footnote{{First In First Out},Premier rentré premier sorti} de
messages, texte ou binaires.Elles implémentent de nombreux mécanismes
pérmétant de garentir la pérénité et la survie des messages. Les
tables sont des conteneurs de données structurées, elle peuvent
contenir de nombreux objet de petite taille.\\



\subsubsection{Azure SQL}
Azure SQL est un serveur SQL hébergé dans des machines proches des
VM. Il sagit en réalité d'un SQL Server légèrement modifié.\\
%%%%%%%%%%%%%%%%%%%%%%%%%%%%%%%%%%%%%%%%%%%%%%%%%%%%%%%%%%%%%%%%%%%%%%%
%
%  Archi globale
%
%%%%%%%%%%%%%%%%%%%%%%%%%%%%%%%%%%%%%%%%%%%%%%%%%%%%%%%%%%%%%%%%%%%%%%%


\subsection{Architecture globale}
Pour accompagner la construction du projet au niveau architecture
logicielle deux réunions ont étées organisées avec des participants
extérieurs à \maxsea. La première avec des responsables du
\textit{MCIA}\,\footnote{\textit{Mésocentre de Calcul Intensif
    Aquitain}} à surtout aboutit sur des considérations théorique sur
le cloud computing. Ces personnes ont davantage l'habitude de
travailler sur des offres \textit{IaaS} et sur des environnement
UNIX. La seconde fut à Paris avec un expèrs en architecture Azure de
Mircosoft France et deux employées de la société YSANCE, qui ont fait
un projet semblable au notre. Cette dernière fut plus concrète car les
participants avaient l'habitude de manipuler les API de Windows
Azure. \\




La solution apportée consiste en un déployement et donc en une
solution C\# Visual Studio\,\footnote{\textit{IDE} (Integrated development
  Environment) de Mircosoft.}.
 %% WARGING !!  Besoin de le dire? Première fois que ça apparait ?




\subsubsection{La classe \textit{WorkerRole}}
La classe \textit{Worker Role} constitue le point d'entré du
programme, elle hérite d'ailleurs de \textit{RoleEntryPoint}. Celle ci
va instancier n threads de travail, puis va entrer dans une boucle
infinie de surveillance de ces threads. La création de ces threads ce
fait grace à la classe \textit{Thread}. Ils instancient une classe
\textit{Worker}. La surveillance n'a pas était implémentée, elle
consiste à vérifier qu'il n'y a pas de fuite mémoire, pour au besoin
stopper ou relancer des threads de travail. 


\subsubsection{La classe \textit{Worker}}
La classe textit{Worker} sert à trouver et réépartir les tâches de
travail. Dans son instanciation elle créé dynamiquement, en utilisant
le .............. , un tableau de d'objet héritant de la classe
abstraite \textit{BusinessProcess} et implémentant l'intérface
\textit{IBusinessProcess}. Cette intérface demande d'implémenter deux
méthodes, à savoir \textit{Init()} et \textit{Work()}. La seconde
renvoie $false$ si l'objet ne trouve pas de travail et $true$ après
avoir traité une tâche sinon. L'objet \textit{Worker} bouclera sur le
tableau pour que toutes les tâches soit traitées.


\subsubsection{Les classes héritant de \textit{BusinessProcess}}
Celles ci se comportent globalement de la même manière. Lors de son
instanciation ou de l'appel à \textit{Init()} un objet héritant de
\textit{BusinessProcess} met en place son environement de travail, et
crée notamment des chaines de connexion. Ensuite lors de l'appel à la
méthode \textit{Work} il vérifit si il a une tâche à faire. La plupart
du temps la liste des tâche est implémenté grace à un objet
\textit{CloudQueue} de l'API Azure. \\

Si une tâche est trouvée alors la commande est interprétée à l'aide de
la classe \textit{CommandLine} et les fichiers nécessaires sont
ramenés des blobs Azure sur la VM (en local) afin de travaillé avec.\\

Une fois cette étape de mise en place terminée, le code métier est
appelé. Il écrit soit directement les résultats dans une table SQL
Azure, soit dans un fichier qui sera ensuite remis dans un blob Azure.
Enfin le message correspondant à la tâche est supprimé.


\subsubsection{La classe \textit{Manager}}

Un autre Worker Role nommé \textit{Manager} sert à générer les
milliers de messages envoyés aux queues de travail. C'est un processus
similaire au premier. \\

La classe \textit{WorkerRole} sert de point d'entrée, mais cette fois
il n'y a pas de thread de travail et de thread de
surveillance. L'objet instancira directement un tableau de d'objet
héritant de \textit{Manager} et implémentant l'intérface
\textit{IManager}. Il boucle ensuite sur ce tableau pour savoir si il
existe des taches à effectuer. Ces taches sont différentes des
précédente puisqu'il s'agit de créer des messages et de remplire des
queux.


%\subsubsection{}







%%%%%%%%%%%%%%%%%%%%%%%%%%%%%%%%%%%%%%%%%%%%%%%%%%%%%%%%%%%%%%%%%%%%%%%
%
% Choix d'implé
%
%%%%%%%%%%%%%%%%%%%%%%%%%%%%%%%%%%%%%%%%%%%%%%%%%%%%%%%%%%%%%%%%%%%%%%%

\subsection{Détail de l'architecture et choix d'implémentation}
Nous allons ici aborder les différents choix technique
d'implémentation. %Point par point.

\subsubsection{}

\subsubsection{}

\subsubsection{}

\subsubsection{}
