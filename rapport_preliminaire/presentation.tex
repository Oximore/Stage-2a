% intro.tex

%présentation de l'entreprise ?
\subsection{Présentation de \maxsea}
\textit{Maxsea International} est une entreprise de production de
logiciels maritimes. Son produit phare "Maxsea Time Zero" est un
logiciel d'aide à la navigation maritime. Sa filiale \textit{MapMedia}
produit les cartes lisibles par le logiciel. Bien que son siège social
soit à Bidart, sur la côte Basque, elle possède une filiale en Espagne
et une autre aux États-Unis.


%les coordonnées de son maître de stage en entreprise ou en
%laboratoire d'accueil,
\subsection{Contacts}

\subsubsection*{Entreprise : \maxsea}
\begin{itemize}
\item Adresse : Technopole Izarbel, 64210 Bidart – France
\item mail : info@maxsea.fr
\item tel : + 33 559 43 81 00
\item fax : + 33 559 43 81 01
\end{itemize}

\subsubsection*{Maitre de Stage : Ronan GOLHEN}
\begin{itemize}
\item tel : + 33 614 17 65 05
\item mail : ronan.golhen@maxsea.fr
\end{itemize}


\subsubsection*{Tuteurs de stages}
\textbf{Arnaud CAPDEVIELLE}
\begin{itemize}
\item mail : arnaud.capdevielle@maxsea.fr 
\end{itemize}


\textbf{Arnaud REMY}
\begin{itemize}
\item mail : arnaud.remy@maxsea.fr
\end{itemize}



\subsection{Description du stage}
%le titre du sujet et les mots-clés,
\subsubsection*{Titre :} 
{\centering Migration d'une chaine de production cartographique
  vers un Cloud. \\}

\subsubsection*{Mots-clés :}
{\centering Cloud computing, Production cartographique, Windows Azure,
  Géomatique, Calcul parallèle. \\}

\subsubsection*{Sujet du stage :}
Dans le cadre de la production de données cartographiques vecteur, la
société \maxsea veut employer les technologies "Cloud" et en
particulier la solution "Azure" de \textit{Microsoft}. Pour cela, il
s’agira de coder des logiciels en $C\#$, s’appuyant sur le Framework
open source \textit{GDal/OGR}\,\footnote{\textit{Geospatial Data
    Abstraction Librairy} license by the \textit{Open Source
    Geospatial Foundation} (\textit{OSGeo})} (OGR pour le traitement
de données vecteur). Ceux-ci traiteront des étapes de la production de
données. Leur intérêt est la forte montée en charge possible dans le
Cloud, afin de réduire les temps de production (2 lots de production
par an, couvrant l’ensemble des cartes marines du globe).

\subsubsection*{Dates de stage :}
Le stage commence le 6 juin 2012 et finit le 21 septembre 2012.
