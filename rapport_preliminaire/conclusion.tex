% conclusion.tex

% Ensuite, mensuellement, l'étudiant fait parvenir un état
%d'avancement par mail à son tuteur-enseignant. En une page maximum,
%il doit décrire l'état d'avancement des travaux et les perspectives
%envisagées.

%Un état des lieux de l'avancement actuel du projet est joint dans le
%mail renfermant ce document.

%\begin{comment}
Après une comparaison de plusieurs offres, notamment celles de
Microsoft, Amazon ainsi que celle du MCIA\,\footnote{Mésocentre de
Calcul Intensif Aquitain}, la solution Azure a été définitivement
adoptée.

Le stagiaire a, tout d'abord, pris connaissance de
l'API\,\footnote{\textit{Application Programming Interface}, Interface
  de programmation} métier OGR en créant une application console
transformant un fichier s57 en une base de données SQLite spatiale.

Puis, il a manipulé les différents objets Azure, spécialement les
Blobs, les Queues et SQL Azure, à travers l'API
\textit{.NET} afin de pouvoir exécuter le précédent programme au sein
du Cloud Microsoft.

Il a ensuite recodé une des opérations géométrique de découpage
(\textit{"clipping"}) de la production, faite précédemment par FME. Il
effectue en ce moment des tests de performance sur cette partie qui
constitue une étape importante dans la montée en charge du
processus. Les permiers tests sont concluant puisque un déploiment de
18 instances de travail sur Azure a réussit à faire le travail d'une
journée en moins de 3h.

Une rencontre avec un expert \textit{Microsoft France} en architecture
logiciel aura lieu le 19 juillet à Paris. Cet échange aura pour
principal objectif de bien structurer l'application pour exploiter au
mieux les atouts d'Azure.


%\end{comment}
