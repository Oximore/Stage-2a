% cahier.tex


%le cahier des charges de son travail (en 5 pages maximum) :

%4 formats de cartes? Les prod mapmedia ne sont pas des dev

% - contexte qui décrit l'existant,
\subsection{Description de l'existant}
Pour produire ses cartes, la société \maxsea reçoit chaque année
diverses livraisons de cartes au format S57. Ses deux principaux
fournisseurs, \textit{Jeppesen} et \textit{Navionics}, livrent deux
fois par an l'ensemble des cartes. Une seule de ces livraisons
contient 45000 fichiers. Au début du stage la chaine de production des
cartes au format dbv était localisé dans les locaux de
l'entreprise. Le temps de transformation était cependant préoccupant,
puisque les livraisons prenaient du retard. En effet, en utilisant les
12 serveurs disponibles pour cette opération de production ,quatre
mois étaient nécessaires au traitement d'une livraison.

La chaine de production de cartes vecteurs est supervisée par le
logiciel SSIS\,\footnote{SQL Server Integration Services} de
\textit{Microsoft}. Une fois les données reçues, par DVD ou
FTP\,\footnote{\textit{File Transfert Protocol}, Protocole de
  transfert de fichiers}, un opérateur lance l'exécution de
l'ETL\,\footnote{\textit{Extract Transform Load}, extraction
  transformation alimentation} SSIS. Celui ci va lancer un par un les
différents traitements qui seront nécessaires à la
production. L'opérateur effectue différents contrôles plus ou moins
automatisés, notamment à la fin de la création du catalogue. Durant ce
processes les cartes sont manipulées sous forme de bases de données
spatiales. Les opérations sur celles-ci se font principalement via
FME\,\footnote{Logiciel propriétaire de SAFE SOFTWARE}, un ETL
Spatial, qui s'appuit sur le Framework open source spatial
\textit{GDal/OGR}.
%ici un schéma des étapes de production

% - description de la mission technique (travail à effectuer dans le
% contexte de l'étude)
\subsection{Mission Technique}
Le stage consiste à déployer la chaine de production dans un
environement de cloud computing. Ceci doit amener trois avantages à la
société. Le premier et le plus important est de pouvoir produire les
cartes dans le temps demandé. Ensuite vient la libération de
ressources machine au sein des locaux et la baisse du prix de revient
de la chaine de production vecteur.

Les pistes explorées par Ronan GOLHEN mettent en avant l'offre de
\textit{PaaS}\,\footnote{\textit{Plateform as a Service}, Plateforme
  en tant que service} Azure de \textit{Microsoft}. Les
caractéristiques principales de cette offre sont de libérer la société
de la gestion des serveurs et de ne payer qu'à l'usage. Sa philosophie
incite fortement aux modèles de calcul parallèle.

Le stagiaire devra mettre en place une architecture logicielle
permettant de traiter une partie ou l'ensemble de la production
vecteur à l'intérieur du Cloud \textit{Microsoft}. Ce travail devra
être exploitable par les opérateurs de \textit{MapMedia} sans
apprentissage des technologies de developpement.




%- moyens à mettre en œuvre.
\subsection{Moyens à mettre en œuvre}
Migrer la chaine de production demande une bonne connaissance des
principes de workflow, de développement Azure et du Framework métier
\textit{OGR}.

 Le workflow est à reconsidérer, on ne peut pas juste copier la chaine
 existante pour deux raisons principales. Tout d'abord le service
 hébergeur est distant et on ne peut pas se permettre de controler
 toute la production dans les locaux, ce qui impliquerait un
 rapatriment des données entre chacune de ces étapes. Ensuite le
 fait que la chaine utilise des logiciels propriétaires pose un
 problème de licence car ils sont liés à une machine physique,
 principe à bannir en terme de technologie cloud.

Azure possède un Framework particulier qu'il faut maitriser. Par
exemple les espaces de stockage permanent sont séparés des ressources
physiques de la machine, qui elles, deviennent des ressources
temporaires.

 \textit{OGR} sera la principale alternative à FME. En effet on doit
 pouvoir se passer de ce logiciel propriétaire et on peut se servir du
 Framework qu'il utilise puisque ce dernier est open source. Il faut
 donc recoder les étapes de production sans passer par l'intermédiaire
 de FME.



